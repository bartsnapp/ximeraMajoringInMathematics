\documentclass{ximera}

\graphicspath{
	{./}
	{ximeraTutorial/}
	{eulerCharacteristic/}
	{payingOffDebt/}
	{drugDecay/}
	{sphericalGeometry}
	{inflation}
	{division}
	}

\newenvironment{sectionOutcomes}{}{}

\usepackage{pgfplots}
%\pgfplotsset{compat=1.15}
\usepackage{mathrsfs}
\usetikzlibrary{arrows}


%\usepackage{wasysym}
%\newcommand{\pt}{\text{\smiley{}}}
\newcommand{\pt}{\bigstar}
\usepackage{tkz-euclide}
\usetikzlibrary{backgrounds} %% for boxes around graphs
\usetikzlibrary{shapes,positioning}  %% Clouds and stars
%\usetkzobj{all}
\usepackage{multicol}



\title{Drug Decay, Part 2}
\author{Bart Snapp \and Brad Findell}
\begin{document}
\begin{abstract}
Here we investigate various models for the decay of drugs in our bodies.  
\end{abstract}
\maketitle


Pharmacokinetics is the study of the behavior of drugs and other substances administered to living organisms.  
The well-established general model for 
the elimination of a given drug from the body of a given human being is given by the differential equation 
$$y'(t)=-\frac{ky(t)}{A+y(t)}$$
where $y(0)$ is the initial concentration in, say, milligrams per liter, and $y(t)$ is the concentration at time $t$.  The constants $k$ and $A$ depend, of course, on the particular drug and the particular human being. 


\begin{question}
When the drug in question is alcohol, $y(t)$ is usually rather large in comparison to $A$.   With this assumption, we can approximate the general model with a simpler model, $y'(t)=\answer{-k}$.    

If $C=y(0)$, the initial concentration at time $t=0$, then the general solution to this differential equation is 
\[  
y(t) =\answer{ -kt + C}.  
\]
\end{question}

% Analyze various Blood Alcohol Content (BAC) calculators
%
 
\begin{question}

As you know, alcohol consumption can be dangerous.  Even if you don't drink, you probably have friends and neighbors who do, so some knowledge might be worthwhile.  

People drinking alcohol often monitor their consumption by counting ``drinks,'' (i.e., servings).  Informally, one drink is a bottle of beer, a glass of wine, or a shot of hard liquor.  So that each serving has the same amount of alcohol, these informal descriptions are standardized as follows:  

\begin{itemize}
\item One $\answer{12}$ ounce can of beer at $\answer{5}$ percent alcohol. 
\item One $\answer{5}$ ounce glass of wine at $\answer{12}$ percent alcohol. 
\item One $\answer{1.5}$ ounce shot of hard liquor at $\answer{40}$ percent alcohol.  % 200 proof
\end{itemize}

Observe that all of these have $\answer{0.6}$ ounces of alcohol.  Craft beer often has higher alcohol content and sometimes comes in larger cans.  For example, a $16$ ounce India pale ale at $7.5\%$ actually contains $\answer{16\times 0.075}$ ounces of alcohol, which would be more like $\answer{16\times 0.075\div 0.6}$ drinks.  Mixed drinks are sometimes made with one shot of hard liquor, but check with the bartender.  

\begin{question}

Since 1979 in the U.S., standard wine and liquor bottles have been 750 milliliters.  A quart is one fourth of a gallon.  But these bottles are called \emph{fifths} because the standard size used to be one fifth of a gallon.  

Note: Here are some useful conversions:  $1 \textrm{ quart} = 32 \textrm{ fl.\ oz.}$;  $1 \textrm{ gallon} \approx 3.785 \textrm{ liters}$.  

Question: How close is the metric ``fifth'' to  a fifth of gallon? 
 
Answer: One fifth of a gallon is $\answer{3.785\div 5}$ milliliters.  So the 750 milliliter bottle is $\answer[tolerance=0.5]{3.785\div 5 - 750}$ milliliters 
\wordChoice{\choice[correct]{less}\choice{more}}.  

Moreover, 750 milliliters = $\answer{750\div 3785 \times 32 \times 4}$ ounces, which would be $\answer[tolerance=0.055]{750\div 3785 \times 32 \times 4 \div 5}$ ``glasses'' of wine, or $\answer[tolerance=0.055]{750\div 3785 \times 32 \times 4 \div 1.5}$ shots of liquor.  (Answer to the closest tenth of a glass of wine or tenth of a shot.) 


\end{question}
\end{question}



% Question about time of elimination, graphically 
\begin{question}
\end{question}


Again, the general model for  the elimination of a given drug from the body of a given human being is given by the differential equation 
$$y'(t)=-\frac{ky(t)}{A+y(t)}$$
where $y(0)$ is the initial concentration in, say, milligrams per liter, and $y(t)$ is the concentration at time $t$. 

\begin{question}
When the drug in question is an over-the counter pill, such as ib

, $y(t)$ is usually very small in comparison to $A$.   With this assumption, we can approximate the general model with a simpler model, $y'(t)=\answer{-\frac{k}{A}}y(t)$.    

If $C=y(0)$, the initial concentration at time $t=0$, then the general solution to this differential equation is 
\[  
y(t) =\answer{Ce^{-\frac{k}{A}t}}. 
\]

\end{question}

\begin{question}
When we connect the dots, should the connection be linear, concave up, or concave down? 

Should the 

\end{question}


\end{document}
