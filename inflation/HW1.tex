\documentclass{ximera}

\graphicspath{
	{./}
	{ximeraTutorial/}
	{eulerCharacteristic/}
	{payingOffDebt/}
	{drugDecay/}
	{sphericalGeometry}
	{inflation}
	{division}
	}

\newenvironment{sectionOutcomes}{}{}

\usepackage{pgfplots}
%\pgfplotsset{compat=1.15}
\usepackage{mathrsfs}
\usetikzlibrary{arrows}


%\usepackage{wasysym}
%\newcommand{\pt}{\text{\smiley{}}}
\newcommand{\pt}{\bigstar}
\usepackage{tkz-euclide}
\usetikzlibrary{backgrounds} %% for boxes around graphs
\usetikzlibrary{shapes,positioning}  %% Clouds and stars
%\usetkzobj{all}
\usepackage{multicol}



\title{Inflation Ideas}
\author{Bart Snapp \and Brad Findell}
\begin{document}
\begin{abstract}
Here we explore the concept of inflation using recent data from the U.S.
\end{abstract}
\maketitle

The aim of this homework is to build tools to explain the concept inflation and what it tells us about changing prices in the U.S. economy.  

Describe what happened to prices from January 2020 (just before the COVID-19 pandemic) through the present.  

\vspace{0.15in}

Here are some sentences that you may borrow or adapt in your explanation:  
\begin{itemize}
\item The U.S. Bureau of Labor Statistics (BLS) tracks the prices of many goods and services in the U.S. economy.  In addition, it also compiles monthly data to compute the Consumer Price Index (CPI), which is a weighted average of a hypothetical ``market basket'' of goods purchased by typical U.S. households.  
\item Inflation can be calculated over any time period, but to compare inflation over different periods of time, it helps to compare annualized rates, just 
as loans are usually described with an ``annual percentage rate'' (APR). 
\item Prices of some goods and services (such as gasoline and some foods) are ``volatile,'' meaning they fluctuate monthly, weekly, or even daily.  Prices of other goods and services (such as wages and rent) are described as ``sticky,'' meaning they respond slowly to changing economic conditions. 
\item The Federal Reserve System (the Fed) uses $2\%$ (annual) core inflation as a ``target'' to guide its policy decisions.   Core inflation excludes food and energy prices because of their volatility.  
\end{itemize}


To get started, download data from the Consumer Price Index (\href{https://fred.stlouisfed.org/series/CPIAUCSL}{https://fred.stlouisfed.org/series/CPIAUCSL}) and create a spreadsheet to support computations.  You might find it useful for your paper to include answers to some of the following questions: 

\begin{enumerate}
\item What is the rate of inflation from September 2024 to October 2024?  What is that rate annualized?  
\item What are the maximum and minimum monthly inflation rates between January 2020 (just before the pandemic) and October 2024?  When do they occur and what are those rates annualized?  
\item Use year-over-year inflation rates to describe the ``inflation trajectory'' from just before the pandemic to the present.  (A graph may help.) Be sure to mention the maximum and minimum rates and when they occur.  Why are these year-over-year rates cited in the press more often than the annualized monthly rates?  
\item How does inflation now compare to inflation at its peak since 2020.  How close are we to reaching the Fed's target rate? Explain.  
\item What is the total inflation from January 2020 and October 2024?  What is that rate annualized?   Explain. 
\item Use your data and calculations to explain and illustrate the difference between ``deflation'' and ``disinflation.''
\end{enumerate}


With slight inflation, debt is manageable.     

For example, when inflation is $2\%$, the ``real'' cost of a $5\%$ mortgage is only $3\%$.  Many consumers stretch to afford their first mortgage.  But with raises and promotions (and a little inflation), the mortgage becomes more affordable after $5$ or $10$ years.  

Study the graph above to answer the following equations

During which year(s) was there \emph{deflation}?  An economy experiences deflation when overall prices go down.  

Slight month-over-month deflation is not uncommon due to overlapping factors, including the following: 

(1) Prices of most goods and services change only slightly, if at all. 
(2) Prices of other goods (e.g., food and energy) are quite volatile.  For example, 

 and the process involves inherent measurement error.  

But significant, prolonged deflation is undesirable. Falling prices are a sign of a weak economy.  Consumers delay purchases in hopes of getting a lower price in the future.  Companies slow production in response to lower demand, and eventually resort to layoffs or wage reductions, aiming to maintain profitability.  The real value of consumer and corporate debt increases, making it harder to pay off.  Unemployment rises, which further reduces demand and, as a consequence, prices.  



``real value'' means adjusted for inflation.  

Typically, deflation is a sign of a weakening economy. Economists fear deflation because falling prices lead to lower consumer spending, which is a major component of economic growth. Companies respond to falling prices by slowing down their production, which leads to layoffs and salary reductions. This further lowers demand and prices.




\end{document}
