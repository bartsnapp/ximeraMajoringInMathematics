\documentclass[handout,space,nooutcomes]{ximera}
%\documentclass{ximera}

\graphicspath{
	{./}
	{ximeraTutorial/}
	{eulerCharacteristic/}
	{payingOffDebt/}
	{drugDecay/}
	{sphericalGeometry}
	{inflation}
	{division}
	}

\newenvironment{sectionOutcomes}{}{}

\usepackage{pgfplots}
%\pgfplotsset{compat=1.15}
\usepackage{mathrsfs}
\usetikzlibrary{arrows}


%\usepackage{wasysym}
%\newcommand{\pt}{\text{\smiley{}}}
\newcommand{\pt}{\bigstar}
\usepackage{tkz-euclide}
\usetikzlibrary{backgrounds} %% for boxes around graphs
\usetikzlibrary{shapes,positioning}  %% Clouds and stars
%\usetkzobj{all}
\usepackage{multicol}


%\usepackage[strict]{changepage}

\title{GeoGebra Commands}
\author{Brad Findell}
\begin{document}
\begin{abstract}
Summaries of GeoGebra Commands for Financial Math
\end{abstract}
\maketitle


$Payment( <\text{Rate}>, <\text{Number of Periods}>, <Present Value>, <Future Value (optional)>, <Type (optional)> )$
Calculates the payment for a loan based on constant payments and a constant interest rate.

FutureValue( <Rate>, <Number of Periods>, <Payment>, <Present Value (optional)>, <Type (optional)> )
Returns the future value of an investment based on periodic, constant payments and a constant interest rate.

Periods( <Rate>, <Payment>, <Present Value>, <Future Value (optional)>, <Type (optional)> )
Returns the number of periods for an annuity based on periodic, fixed payments and a fixed interest rate.

PresentValue( <Rate>, <Number of Periods>, <Payment>, <Future Value (optional)>, <Type (optional)> )
Returns the total amount of payments of an investment.

Rate( <Number of Periods>, <Payment>, <Present Value>, <Future Value (optional)>, <Type (optional)>, <Guess (optional)> )
Returns the interest rate per period of an annuity.

\begin{itemize}
\item <Future Value (optional)> A cash balance you want to attain after the last payment. If you do not enter a future value, it is assumed to be 0.
\item <Guess (optional)> Your guess for what the rate will be.
\item <Number of Periods> Total number of payments for the loan.
\item <Payment> The amount paid in each period.
\item <Present Value (optional)> Total amount that a series of future payments is worth now. If you do not enter a value, it is assumed to be 0.
\item <Present Value> Total amount that a series of future payments is worth now.
\item <Rate> Interest rate per period.
\item <Type (optional)> Indicates when payments are due. If you do not enter a value or you enter 0 the payment is due at the end of the period. If you enter 1 it is due at the beginning of the period.
\end{itemize}

\begin{itemize}
\item Note: Make sure that you are consistent about the units you use for <Rate> and <Number of Periods>. If you make monthly payments on a four-year loan at an annual interest rate of 12 percent, use 12\%/12 for rate and 4*12 for number of payments.

\item Note: For all arguments, cash paid out is represented by negative numbers and cash received by positive numbers.
\end{itemize}

\end{document}
