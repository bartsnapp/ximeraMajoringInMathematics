\documentclass{ximera}

\outcome{Understanding working with units.}
\outcome{Understanding the magnitude of measurments.}
\outcome{Understanding relevant measurments.}


\title{A Beautiful Moon}

\begin{document}
\begin{abstract}
Here we investigate what would happen if the Moon were as close as the
International Space Station.
\end{abstract}
\maketitle

\section{A closer Moon}

Orbits of planets (and other things) in space basically follow paths
determined by conic sections. A conic section is a ``slice'' of a
cone. Currently, there is a space station orbiting the Earth at an average
altitude of $420$ kilometers.

\begin{problem}
Put this distance into perspective by comparing this distance to
``landmarks'' on the Earth.
\begin{freeResponse}
\end{freeResponse}
\end{problem}

The Moon is around $384400$ kilometers from the Earth (measured from
the center of the Moon to the center of the Earth). Now suppose that
the Moon was the same distance as the International Space Station, as
measured from the surface of the Moon to the surface of the Earth:
\youtube{http://www.youtube.com/watch?v=oBDZtt0vWD8}

\begin{center}
\textbf{How amazing and terrifying!}
\end{center}

\break

\begin{problem}
Play around with this idea for a bit. Come up with as many questions
as you can and list them.
\begin{freeResponse}
\end{freeResponse}
\end{problem}

\begin{problem}
As a class, decided one (or more) of these problems to
investigate. Reflect upon the investigation and give a solution (if
you have one).
\begin{freeResponse}
\end{freeResponse}
\end{problem}


%% \begin{question}
%% What would the gravitational effect of this closer Moon be? Let's hear
%% some speculation!
%% \begin{freeResponse}
%% Answers will vary. 
%% \end{freeResponse}
%% \end{question}

%% \begin{question}
%% What sort of answer should we be looking for? Without attempting to
%% compute the answer, tell us everything you can about it.
%% \begin{freeResponse}
%% Answers will vary. 
%% \end{freeResponse}
%% \end{question}

%% We can start to figure this out with some physics and mathematics.
%% Newton's \textit{law of universal gravitation} states that the
%% magnitude of the force given by two masses is
%% \[
%% F = G \cdot \frac{m\cdot M}{r^2} 
%% \]
%% where $F$ is measured in newtons, $m$ and $M$ are in kilograms, $r$ is
%% in meters, and the constant $G$ is approximately $6.672\times
%% 10^{-11}\, \mathrm{N}\, \mathrm{m}^2/\mathrm{kg}^2$. On the other hand
%% the acceleration due to gravity on the Earth's surface is around
%% $9.8\, \mathrm{m}/\mathrm{s}^2$, Newton's second law of motion states:
%% \[
%% F = ma
%% \]
%% where $F$ is the magnitude of the force in newtons.


%% The Moon is around...





%% \video{http://youtu.be/oBDZtt0vWD8}

%% \begin{question}
%% Would the Moon's gravity pull us off into space? 
%% \end{question}

%% \section{The Roche limit} %http://en.wikipedia.org/wiki/Roche_limit

\end{document}
