\documentclass{ximera}

\outcome{Understanding working with units.}
\outcome{Understanding the magnitude of measurments.}
\outcome{Understanding relevant measurments.}

\title{A Beautiful Moon}

\begin{document}
\begin{abstract}
Here we investigate what would happen if the Moon were as close as the
International Space Station.
\end{abstract}
\maketitle

\section{A closer Moon}

Orbits of planets (and other things) in space basically follow paths
determined by conic sections. A conic section is a ``slice'' of a
cone. Currently, there is a space station orbiting the Earth at an average
altitude of $250$ kilometers.

\begin{question}
Put this distance into perspective by comparing this distance to
``landmarks'' on the Earth.
\begin{solution}
\begin{freeResponse}
Consider this, $250$ kilometers is around $155$ miles. Indianapolis,
Indiana is around $175$ miles directly away from Columbus, Ohio. On
the other hand, Lexington, Kentucky is almost exactly $155$ miles away
as the crow-flies and so are Pittsburgh, Pennsylvania, and Detroit,
Michigan.
\end{freeResponse}
\end{solution}
\end{question}


Newton's \textit{law of universal graviation} states that the
magnitude of the force given by two masses is
\[
F = G \cdot \frac{m\cdot M}{r^2} 
\]
where $F$ is measured in netwons, $m$ and $M$ are in kilograms, $r$ is
in meters, and the constant $G$ is approximately $6.672\times
10^{-11}\, \mathrm{N}\, \mathrm{m}^2/\mathrm{kg}^2$.

The Moon is around...


Now suppose that the Moon was the same distance as the International
Space Station\youtube{http://youtu.be/oBDZtt0vWD8}. How amazing and terrifying!

\begin{question}
Would the Moon's gravity pull us off into space? 
\end{question}

\section{The Roche limit} %http://en.wikipedia.org/wiki/Roche_limit

\end{document}
