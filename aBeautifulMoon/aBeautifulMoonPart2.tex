\documentclass[handout,nonumbers,space]{ximera}

\outcome{Understanding working with units.}
\outcome{Understanding the magnitude of measurments.}
\outcome{Understanding relevant measurments.}

\title{A Beautiful Moon}

\begin{document}
\begin{abstract}
We further investigate what would happen if the Moon were as close as the
International Space Station.
\end{abstract}
\maketitle
Remember this video?

\youtube{http://www.youtube.com/watch?v=oBDZtt0vWD8}

\begin{problem}
What questions do we have concerning gravity and this situation? Solve
as many as you can.
\begin{freeResponse}
\end{freeResponse}
\end{problem}

\begin{problem}
What questions do we have about orbital speed and this situation?
Solve as many as you can.
\begin{freeResponse}
\end{freeResponse}
\end{problem}


%% \begin{question}
%% What would the gravitational effect of this closer Moon be? Let's hear
%% some speculation!
%% \begin{freeResponse}
%% Answers will vary. 
%% \end{freeResponse}
%% \end{question}

%% \begin{question}
%% What sort of answer should we be looking for? Without attempting to
%% compute the answer, tell us everything you can about it.
%% \begin{freeResponse}
%% Answers will vary. 
%% \end{freeResponse}
%% \end{question}

%% We can start to figure this out with some physics and mathematics.
%% Newton's \textit{law of universal gravitation} states that the
%% magnitude of the force given by two masses is
%% \[
%% F = G \cdot \frac{m\cdot M}{r^2} 
%% \]
%% where $F$ is measured in newtons, $m$ and $M$ are in kilograms, $r$ is
%% in meters, and the constant $G$ is approximately $6.672\times
%% 10^{-11}\, \mathrm{N}\, \mathrm{m}^2/\mathrm{kg}^2$. On the other hand
%% the acceleration due to gravity on the Earth's surface is around
%% $9.8\, \mathrm{m}/\mathrm{s}^2$, Newton's second law of motion states:
%% \[
%% F = ma
%% \]
%% where $F$ is the magnitude of the force in newtons.


%% The Moon is around...





%% \video{http://youtu.be/oBDZtt0vWD8}

%% \begin{question}
%% Would the Moon's gravity pull us off into space? 
%% \end{question}

%% \section{The Roche limit} %http://en.wikipedia.org/wiki/Roche_limit

\end{document}
