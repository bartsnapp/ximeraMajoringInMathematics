%\documentclass[handout,space,nooutcomes]{ximera}
\documentclass{ximera}

\graphicspath{{./}{eulerCharacteristic}}

%\usepackage[strict]{changepage}


\title{Spherical Geometry}
\author{Brad Findell \and Bart Snapp}
\begin{document}
\begin{abstract}
Here we investigate spherical geometry and compare to Euclidean (flat) geometry. 
\end{abstract}
\maketitle


\begin{question}
Among ``surfaces of constant curvature,'' there are three types:
\begin{itemize}
\item Elliptic surfaces (such as a sphere) have positive curvature.
\item Euclidean (flat or ``flattenable'') surfaces have zero curvature.
\item Hyperbolic surfaces have negative curvature.
\end{itemize}
Explore curvature in GeoGebra \href{https://www.geogebra.org/m/qgyhqgt9}{here}. 

See models of hyperbolic surfaces \href{https://www.theiff.org/oexhibits/oe1e.html}{here}.

For more, search for ``crochet hyperbolic plane.''
\end{question}

\begin{question}
Representing curved surfaces on a flat computer screen requires some distortion.

We now see a disc model of spherical geometry and a Poincare disc model of hyperbolic
geometry. Both models represent ``straight'' lines with curves, and both models represent angles accurately.
(Other models make different choices.)

What happens to spherical or hyperbolic geometry as the radius of the disc gets large?

\end{question}


\begin{question}
In the Poincare disc model of hyperbolic geometry, what did you notice about the sum of the interior angles in a triangle? 

And what happens as the triangle gets larger or smaller?
\end{question}


\begin{question}
Are there parallel lines in hyperbolic geometry?
\end{question}

\begin{question}
In Euclidean (flat) geometry, given three distinct points on a line, it is always
the case that one of the points is between the other two.

Does this work in hyperbolic geometry?
\end{question}


\begin{question}
What can we say about the sum of the interior angles of a quadrilateral in hyperbolic geometry?
\end{question}

\begin{question}
In Euclidean geometry, we think of area as covering with unit squares (including partial squares).

Does this approach work in hyperbolic geometry?
\end{question}

\begin{question}
In Euclidean geometry, parallel lines are always the same distance apart. 
And you can create parallel lines by constructing two perpendiculars to the same line.

What happens to two perpendiculars to the same line in spherical geometry?
\end{question}

\begin{question}
What happens to two perpendiculars to the same line in hyperbolic geometry?
\end{question}

\begin{question}
Draw perpendiculars to the same line in a disc model of (hemi-)spherical geometry.

Hint: In our disc model of (hemi-)spherical geometry, ``lines'' intersect the
boundary at opposite ends of a diameter of the disc.
\end{question}

\begin{question}
Draw perpendiculars to the same line in a Poincare disc model of hyperbolic geometry.

Hint 1: In the Poincare disc model, ``lines'' intersect the boundary of the disc at right angles.
\end{question}



\end{document}
