\documentclass[nooutcomes]{ximera}

\graphicspath{{./}{eulerCharacteristic}}


\title{Sums and products combined, oh my!}

\begin{document}
\begin{abstract}
Here we investigate a new way to combine numbers.
\end{abstract}
\maketitle

\paragraph{The setting:}

Suppose you wrote the numbers $1,2,3,\dots,100$ on a chalkboard. You
may take any two numbers, erase them, and then add the sum of their
sum and their product to the list.

\begin{problem}
     Write down as many mathematical questions as you can for this
     setting. After you have your questions, label them as ``Level
     1,'' ``Level 2,'' or ``Level 3'' where:
\begin{description}
\item[Level 1] Means you know the answer, or know exactly how to do
  this problem.
\item[Level 2] Means you think you know how to do the problem.
\item[Level 3] Means you have no idea how to do the problem.
\end{description}
\begin{freeResponse}
  Here are some example questions:
  \paragraph{Level 1}
  \begin{enumerate}
    \item Given two positive numbers, if we perform the process on
      these numbers will the result always be larger than the initial
      two numbers?
    \item If we do this process with a number and zero, what will be
      the result?
    \item Given a list of $2$ numbers, after performing the processes
      once, how many numbers are in the list?
    \item Given a list of $100$ numbers, after performing the process
      once, how many numbers are in the list?
    \item If we perform the process enough times, will we always end
      with one number?
    \item Given a list of $100$ numbers, how many times do we need to
      perform the process to end with exactly $1$ number?
    \item Given a list of positive numbers, can we choose which two
      will give us the largest result after performing the process?
    \item Given a list of positive numbers, can we choose which two
      will give us the smallest result after performing the process?
    \item If we think of the process as a binary operation, call it
      $\pt$, is it commutative?
    \item What is $1 \pt 1$?
    \item Are there any two numbers $a$ and $b$ such that $a\pt b = 0$?
    \item Is there a number $I$ such that $a\pt I= a$?
  \end{enumerate}

  \paragraph{Level 2}
  \begin{enumerate}
    \item Given two numbers $a$ and $b$, possibly positive or negative
      or zero, will the result ever be smaller than $a$ or $b$?
    \item If we think of the process as a binary operation, call it
      $\pt$, is it associative?
    \item What does associativity of the operation mean in the context
      of our setting above?
    \item Does the final result change depending on which numbers are
      chosen at each ``turn?''
    \item Are there nonzero numbers $a$ and $b$ with $a\pt b = 0$?
    \item Are there numbers $a$ and $b$ with $a\pt b = 1$?
    \item Are there numbers $a$ and $b$ with $a\pt b = -1$?
    \item What is the final result, if the process is repeated over
      and over, to the list
      \[
      \{1,2,3\}?
      \]
    \item What is the final result, if the process is repeated over
      and over, to the list
      \[
      \{1,1,1\}?
      \]
    \item What is the final result, if the process is repeated over
      and over, to the list
      \[
      \{-1,-1,-1\}?
      \]
    \item What is the final result, if the process is repeated over
      and over, to the list
      \[
      \{1,-1,-1\}?
      \]
  \item What is the final result, if the process is repeated over
      and over, to the list
      \[
      \{1,1,-1\}?
      \]
    \item Can we solve the equation
      \[
      a\pt 5 = 0?
      \]
\end{enumerate}

  \paragraph{Level 3}
  \begin{enumerate}
  \item Given a list of length $n$
    \[
    \{1,1, \dots, 1\}
    \]
    Repeat the process until one number remains. What is this number?
  \item Given a list of length $n$
    \[
    \{2,2, \dots, 2\}
    \]
    Repeat the process until one number remains. What is this number?
  \item Given a list of length $n$
    \[
    \{-1,-1, \dots, -1\}
    \]
    Repeat the process until one number remains. What is this number?
  \item Given a list of length $n$
    \[
    \{-2,-2, \dots, -2\}
    \]
    Repeat the process until one number remains. What is this number?
  \item Given a list of length $n$ of consecutive positive numbers
    Repeat the process until one number remains. What is this number?
  \item Given a list of length $n$ of consecutive nonnegative numbers
    Repeat the process until one number remains. What is this number?
  \item Given a list of length $n$ of consecutive (possibly negative)
    numbers Repeat the process until one number remains. What is this
    number?
  \item Do there exist lists of numbers that will terminate with zero
    when the procedure is applied multiple times?
  \item Describe all lists of length $n$ that end at zero.
  \item Can any numbers \textbf{not} be produced by our procedure?
  \item Can we ``undo'' $\pt$? That is can we solve:
    \[
    x \pt y = n?
    \]
    for both $x$ and $y$?
  \item Given $n$ and $a$, exactly when can we solve the equation
      \[
      a\pt x = n
      \]
      for $x$?
      \item Can we develop an idea of a fraction with $\pt$ by working
        analogously to the relationship between $\times$ and $\div$?
  \item Compute:
    \[
    \frac{d}{dx} (n\star x)
    \]
  \item Compute
    \[
    \int n\star x \,d x
    \]
  \end{enumerate}
\end{freeResponse}
\end{problem}


%% \begin{problem}
%% In class we came up with several questions involing this setting. State three of them. 
%% \begin{freeResponse}
%% \end{freeResponse}
%% \end{problem}

%% \begin{problem}
%%   With this setting, we made up a new operation $\pt$ where
%%   \[
%%   a\pt b = a+b + ab
%%   \]
%%   What is
%%   \[
%%   a\pt b\pt c = \answer{a+b+c+ab+ac+bc+abc}
%%   \]
%% \end{problem}

%% \begin{problem}
%% We showed that $\pt$ is \textbf{commutative}, which means
%% \begin{multipleChoice}
%%   \choice[correct]{$a\pt b = b\pt a$}
%%   \choice{$((a\pt b)\pt c) = a(\pt (b\pt c))$}
%% \end{multipleChoice}
%% \end{problem}

%% \begin{problem}
%% We showed that $\pt$ is \textbf{associative}, which means
%% \begin{multipleChoice}
%%   \choice{$a\pt b = b\pt a$}
%%   \choice[correct]{$((a\pt b)\pt c) = a(\pt (b\pt c))$}
%% \end{multipleChoice}
%% \end{problem}

%% \begin{problem}
%% Finally we encoutered several squences of numbers including
%% \begin{align*}
%%   &\{ 1, 3, 7, 15, 63,\dots\}\\
%%   &\{ 2, 4, 8, 16, 64,\dots\}\\
%%   &\{ 1, 5, 23, 119, 719,\dots\}\\
%%   &\{ 1, 2, 6, 24, 120,\dots\}
%% \end{align*}
%% Look these sequences up at \link{https://oeis.org/}{https://oeis.org/}. Find something intersting at the OEIS and tell us about it.
%% \begin{freeResponse}
%% \end{freeResponse}
%% See you next Tuesday!
%% \end{problem}


\end{document}
