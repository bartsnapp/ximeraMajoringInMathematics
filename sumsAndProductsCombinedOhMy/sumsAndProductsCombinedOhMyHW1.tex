\documentclass{ximera}

\graphicspath{{./}{eulerCharacteristic}}


\title{Sums and Products Homework}
\author{Bart Snapp \and Brad Findell}
\begin{document}
\begin{abstract}
Homework questions about Sums and Products Combined. 
\end{abstract}
\maketitle

\emph{Let's set off the following situation in a box.} 
\begin{quote}
Suppose you wrote the numbers $1,2,3,\dots,100$ on a chalkboard. You
may take any two numbers, erase them, and then write the sum of their
sum and their product as another number in the list.
\end{quote}
Last class, we began investigating this idea, and we generated a lot
of questions.  

\begin{problem}
Suppose in the first turn you remove 6 and 2.  After erasing them, what number will you append to the list?  $\answer{6\times 2+6+2}$
\end{problem}

\begin{problem}
Suppose in the first turn you remove 13 and 5.  After erasing them, what number will you append to the list?  $\answer{13\times 5+13+5}$
\end{problem}

\begin{problem}
The game must end when you have $\answer{1}$ number left.  That means the game will end after $\answer{99}$ turns.
\begin{feedback}[correct]
Correct!  Each turn reduces the length of the list by 1.
\end{feedback}
\end{problem}

\begin{problem}
Perhaps you have noticed that this process defines a new ``operation'' on numbers.  Let's
call the operation $\star$.  Given two numbers $a$ and $b$, $a\star b$
is then ``the sum of their sum and their product.''  In other words, 
\[
a\star b = \answer{a\times b+a+b}
\]
\end{problem}

\begin{problem}
Make the following computations:
\begin{enumerate}
\item $3\star 4= \answer{3\times 4+3+4}$
\item $4\star 5= \answer{4\times 5+4+5}$
\item $4\star 3= \answer{4\times 3+4+3}$
\item $(3\star 4)\star 5 = \answer{19\times 5+5+19}$
\end{enumerate}
\end{problem}


\begin{problem}
Is $\star$ a commutative operation?  Explain.  
\begin{freeResponse}
\end{freeResponse}
\end{problem}


\begin{problem}
Is $\star$ an associative operation?  Explain.  
\begin{freeResponse}
\end{freeResponse}
\end{problem}

\end{document}
