%\documentclass[handout,space,nooutcomes]{ximera}
\documentclass{ximera}

\graphicspath{{./}{eulerCharacteristic}}


\title{Sums and products combined, oh my!  Part 2}
\author{Bart Snapp \and Brad Findell}
\begin{document}
\begin{abstract}
Here we continue to investigate a new way to combine numbers.
\end{abstract}
\maketitle

\emph{Let's set off the following question in a box.} 

Suppose you wrote the numbers $1,2,3,\dots,100$ on a chalkboard. You
may take any two numbers, erase them, and then add the sum of their
sum and their product to the list.

Last class, we began investigating this idea, and we generated a lot of interesting questions.  
In this activity, we address some of these questions.  

\begin{problem}
First note that the process defines a new operation on numbers.  Let's call the operation $\star$.  Given two numbers $a$ and $b$,  $a\star b$ is then ``the sum of their sum and their product.''    In other words, 
\[
a\star b = \answer{a+b+ab}
\]
\end{problem}

\begin{problem}
Is $\star$ a commutative operation?  (Yes or no.) $\answer[format=string]{yes}$

\begin{explanation}
Yes, because $a\star b = \answer{a+b+ab}  = b\star a$.  

\begin{question}
To be more precise, we might write 
\[
a\star b = (a+b)+ab= (b+a)+ba = b\star a,
\]
which demonstrates that the commutativity of $\star$ follows from the commutativity of 
$\answer[format=string]{addition}$ and $\answer[format=string]{multiplication}$. 
\end{question}
\end{explanation}
\end{problem}


\begin{problem}
Is $\star$ an associative operation?  (Yes or no.)  $\answer[format=string]{yes}$

\begin{explanation}
Yes, because 
\[
(a\star b)\star c =  \left(\answer{a+b+ab}\right)\star c = \answer{(a+b+ab)+c + (a+b+ab)c} 
 = a\star \left(\answer{b+c+bc}\right) = a \star(b\star c). 
 \]
%= a + b + c + ab + ac + bc + abc = a+(b+c+bc) + a(b+c+bc) =
\begin{question}
Now that we have established that $\star$ is both associative 
and commutative, the order that we choose numbers \wordChoice{\choice{does} \choice[correct]{does not}} matter.  

For example, 
\begin{align*}
((1\star 2)\star 3 )\star 4) &= (3\star (2\star 1))\star 4) &  \text{by commutativity} \\
                                        &= ((3\star 2)\star 1))\star 4) & \text{by associativity}  \\
                                        &= (4\star (3\star 2)\star 1))) & \text{by associativity}  \\
                                        &= (4\star (2\star 3)\star 1))) & \text{by associativity}  \\
                                        &= (4\star (3\star 2)\star 1))) & \text{by commutativity}  \\
                                        &= ((4\star 3)\star 2)\star 1))) & \text{by associativity},  \\
\end{align*}
which gives some sense of why we can write $1\star 2 \star 3 \star 4$ without parentheses and without ambiguity.  

Thus, in the original situation, if we continue choosing numbers until only one remains, that last number will be 
\[
1\star 2\star 3\star \dots \star 100. 
\]
\end{question}
\end{explanation}
\end{problem}

\begin{problem}
Name an operation that is associative but not commutative.  
\begin{freeResponse}
\begin{hint}
Possible answers: matrix multiplication, function composition.  
\end{hint}
\end{freeResponse}
\end{problem}

\newpage
Back to the situation, let's try some simpler problems.  

\begin{problem}
Try starting with the numbers $1,2,3,\dots, k$ for several small values of $k$.  What do you notice?  

%Beginning with $k = 2$, the pattern is $5, 23, 119, 729, \dots$.  
\[
\begin{array}{c|c}
\text{Beginning List}& \text{Final Value} \\
\hline
1,  2 & \answer{5}  \\
1,  2,  3 & \answer{23} \\  
1,  2,  3 ,  4 & \answer{5!-1} \\
1,  2,  3 ,  4 ,  5 & \answer{6!-1}
\end{array}
\]
\begin{problem}
Try to discern a pattern and make a conjecture about a general formula:  
\[
\begin{array}{c|c}
\text{Beginning List}& \text{Conjectured final Value} \\
\hline
1,  2,  3 ,  \dots, 10 & \answer{11!-1} \\
1,  2,  3 ,  \dots, 100 & \answer{101!-1} \\
1,  2,  3 ,  \dots, n & \answer{(n+1)!-1} 
\end{array}
\]

\begin{hint}
Compare your computed values to known factorials.  The answer boxes can interpret expressions with terms such as $4!$ and the like. 
\end{hint}
\end{problem}
\end{problem}

\begin{problem}
Suppose we change the context.  For example, what if we start with just $1$s?  Try starting with $k$ $1$s, for several small values of $k$.  Make a conjecture to generalize the pattern.    

% Beginning with $k=2$, the pattern is $3, 7, 15, 31, \dots$.  

\[
\begin{array}{c|c}
\text{Beginning List}& \text{Final Value} \\
\hline
1,  1 & \answer{3}  \\
1,  1,  1 & \answer{7} \\  
1,  1,  1 ,  1 & \answer{15} \\
1,  1,  1 ,  1 ,  1 & \answer{31} \\
10 1s & \answer{2^{10}-1} \\
n 1s & \answer{2^n-1} 
\end{array}
\]

\end{problem}

\newpage 

\begin{problem}
Expand the expression $(x+a)(x+b)$.  What is the value of this expression when $x=1$?  What do you notice?

\begin{prompt}
\[
(x+a)(x+b) = x^2+ \left(\answer{a+b}\right)x+\answer{ab}.  
\]
When $x=1$, we get $\answer{1+a+b+ab}$, which is $(a\star b)+\answer{1}$.  
\end{prompt}

\end{problem}

\begin{problem}
Expand the expression $(x+a)(x+b)(x+c)$.  What is the value of this expression when $x=1$?  What do you notice?

\begin{prompt}
\[
(x+a)(x+b)(x+c) = x^3+ \left(\answer{a+b+c}\right)x^2+\left(\answer{ab+bc+ac}\right)x+\answer{abc}.  
\]

When $x=1$, we get $\answer{1+a+b+c+ab+bc+ac+abc}$, which is $1+(a\star b\star c)$
\end{prompt}
\begin{problem}
The \emph{roots} of the polynomial $(x+a)(x+b)(x+c)$ are the solutions to the equation $(x+a)(x+b)(x+c)=0$, which are 
$x=\answer{-a}$, $x=\answer{-b}$, or $x=\answer{-c}$.  

\begin{warning}
For our purposes right now, these negative signs are unnecessary complications.  In a particular problem, when you know the degree of the polynomial and the degree of the term of interest, getting the correct sign is mere bookkeeping.  Worrying about the sign will interfere with a more important goal: talking \emph{generally} about various sums and products of roots.  So in the following discussion we will ignore signs and write all expressions without negative signs.  
\end{warning}

So in this cubic polynomial, the constant coefficient is $(abc)$, which is the $\answer[format=string]{product}$ of the roots.  

And the coefficient of $x^2$ is $(a+b+c)$, which is the $\answer[format=string]{sum}$ of the roots.  

\begin{problem}
Correct!  And the coefficient of $x$ is $\answer{ab+bc+ac}$, which is called \emph{the sum of the roots taken two at a time}.  Note that your expression includes all possible products of two of the roots.  \textbf{Cool Stuff!!}
\end{problem}
\end{problem}
\end{problem}

\newpage 

\begin{problem}
Can you construct algebraic expressions that will help you solve the previous problems?  
\begin{freeResponse}
%Try $(x+1)(x+2)(x+3)\dots(x+k)$ and $(x+1)^k$, again when $x=1$.  
\end{freeResponse}
\vfill
\end{problem}

\begin{problem}
Summarize the results above, reminding us what the problem was and
what the solution is. Explain your reasoning along the way.
\begin{freeResponse}
\end{freeResponse}
\vfill
\end{problem}

\end{document}
