%\documentclass[handout,space,nooutcomes]{ximera}
\documentclass{ximera}

\graphicspath{{./}{eulerCharacteristic}}


\title{Sums and products combined, oh my!  Part 2}
\author{Bart Snapp \and Brad Findell}
\begin{document}
\begin{abstract}
Here we continue to investigate a new way to combine numbers.
\end{abstract}
\maketitle

\emph{Let's set off the following question in a box.} 

Suppose you wrote the numbers $1,2,3,\dots,100$ on a chalkboard. You
may take any two numbers, erase them, and then add the sum of their
sum and their product to the list.

Last class, we began investigating this idea, and we generated a lot of interesting questions.  
In this activity, we address some of these questions.  



First note that the process defines a new operation on numbers.  Let's call the operation $\star$.  Given two numbers $a$ and $b$,  $a\star b$ is then ``the sum of their sum and their product.''    

\begin{problem}
Is $\star$ a commutative operation?  Explain.  
\begin{freeResponse}
$a\star b = a+b+ab= b+a+ba = b\star a$.  
\end{freeResponse}
\vfill
\end{problem}

\begin{problem}
Is $\star$ an associative operation?  Explain.  
\begin{freeResponse}
$(a\star b)\star c = (a+b+ab)\star c = (a+b+ab)+c + (a+b+ab)c 
= a + b + c + ab + ac + bc + abc = a+(b+c+bc) + a(b+c+bc) =
a\star (b+c+bc) = a \star(b\star c)$.  

Note that because $\star$ is both associative 
and commutative, the order that we choose numbers does not matter.  
\end{freeResponse}
\vfill
\end{problem}

\newpage
Back to the situation, let's try some simpler problems.  

\begin{problem}
Try starting with the numbers $1,2,3,\dots, k$ for several small values of $k$.  What do you notice?  
\begin{freeResponse}
Beginning with $k = 2$, the pattern is $5, 23, 119, 729, \dots$.  
\end{freeResponse}
\vfill
\end{problem}

\begin{problem}
What if we start with just $1$s?  Try starting with $k$ $1$s, for several small values of $k$.  What do you notice?  
\begin{freeResponse}
Beginning with $k=2$, the pattern is $3, 7, 15, 31, \dots$.  
\end{freeResponse}
\vfill
\end{problem}

\newpage 

\begin{problem}
Expand the expression $(x+a)(x+b)$.  What is the value of this expression when $x=1$?  What do you notice?
\begin{freeResponse}
$(x+a)(x+b) = x^2+ (a+b)x+ab$.  When $x=1$, we get $1+a+b+ab$, which is $1+(a\star b)$.  
\end{freeResponse}
\vfill
\end{problem}

\begin{problem}
Expand the expression $(x+a)(x+b)(x+c)$.  
\begin{prompt}
$(x+a)(x+b)(x+c) = x^3+ \left(\answer{a+b+c}\right)x^2+\left(\answer{ab+bc+ac}\right)x+\answer{abc}$.  
\end{prompt}

When $x=1$, we get $\answer{1+a+b+c+ab+bc+ac+abc}$, which is $1+(a\star b\star c)$

What is the value of this expression when $x=1$?  What do you notice?
\vfill
\end{problem}

\newpage 

\begin{problem}
Can you construct algebraic expressions that will help you solve the previous problems?  
\begin{freeResponse}
Try $(x+1)(x+2)(x+3)\dots(x+k)$ and $(x+1)^k$, again when $x=1$.  
\end{freeResponse}
\vfill
\end{problem}

\begin{problem}
Summarize the results above, reminding us what the problem was and
what the solution is. Explain your reasoning along the way.
\begin{freeResponse}
\end{freeResponse}
\vfill
\end{problem}

\end{document}
