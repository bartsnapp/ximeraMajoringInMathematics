%\documentclass[handout,space,nooutcomes]{ximera}
\documentclass{ximera}

\title{Division Involving Fractions}

\begin{document}
\begin{abstract}
Here again we explore various ways of thinking about division, and we use those to explain some difficult division situations, such as division by zero and division of fractions.
\end{abstract}
\maketitle


There are many ways to think about division.  Here are three:
\begin{itemize}
\item \textbf{Division as multiplication}  I know that $42\div 6=7$ because $6\times 7=42$.  To figure this out, I asked myself what number could go in the box in $6\times \Box =42$.  In algebra, this is like solving the equation $6x=42$.

\item \textbf{Division as sharing}  To compute $42\div 6$, I can imagine sharing 42 cookies among 6 people, and I can see that each person would get 7 cookies. A more general way to ask the question is to ask, ``If 42 cookies are distributed equally across 6 groups, how many cookies are in one group?''  The answer is ``7 cookies are in one group" because 6 groups of 7 cookies each is 42 cookies in all.  

\item \textbf{Division as measurement}  To compute  $42\div 6$, I can imagine measuring 42 cookies into bags of 6 cookies each.  In other words, I can ask, ``If 42 cookies are distributed equally in groups of 6 cookies each, how many groups can be made?''  The answer is ``7 groups"  because 7 groups of 6 cookies each would be 42 cookies in all.   
\end{itemize}

\begin{problem}
What are some other mathematical or real-world situations that involve division?  Use each one to explain $42\div 6=7$.  
\begin{freeResponse}
\end{freeResponse}
\end{problem}

\begin{problem}
Use one of the above meanings of division to explain the calculations $0\div 2$, $2\div 0$, and $0\div 0$.  
\begin{freeResponse}
\end{freeResponse}
\end{problem}

\begin{problem}
Use another of the above meanings of division to explain the calculations $0\div 2$, $2\div 0$, and $0\div 0$.  
\begin{freeResponse}
\end{freeResponse}
\end{problem}

\begin{problem}
What are some other rules from school mathematics that you would like to be able to explain?  
\begin{freeResponse}
\end{freeResponse}
\end{problem}


\end{document}
