\documentclass[handout]{ximera}

\graphicspath{{./}{eulerCharacteristic}}


\title{Division and Fractions}
\author{Brad Findell \and Bart Snapp}
\begin{document}
\begin{abstract}
Here we explore various ways of thinking about division, and we use those to explain some difficult division situations, such as division of fractions.
\end{abstract}
\maketitle


\begin{problem}
Write down at least three ``real-world'' examples of the dvision problem $12\div 3$.  
\begin{freeResponse}
\end{freeResponse}
\vfill
\end{problem}

The \emph{Common Core State Standards for Mathematics} describes fractions as follows: 

\begin{quote}
3.NF.1. Understand a fraction $1/b$ as the quantity formed by $1$ part when a
whole is partitioned into $b$ equal parts; understand a fraction $a/b$ as
the quantity formed by $a$ parts of size $1/b$.
\end{quote}

\begin{problem}
Write down some questions about this context.  Indicate whether each question is Level 1, 2, or 3.  
\begin{freeResponse}
\end{freeResponse}
\vfill
\end{problem}


\begin{problem}
When dividing by a fraction, why does it work to invert and multiply?
\begin{freeResponse}
\end{freeResponse}
\vfill
\end{problem}

\newpage 

\begin{problem}
When we add and subtract fractions, we use a common denominator.  For dividing fractions, would it be okay to convert to common denominators and then just divide the numerators?  Explain why or why not. 
\begin{freeResponse}
\end{freeResponse}
\vfill
\end{problem}


\begin{problem}
When we multiply fractions, we multiply numerators and denominators straight across.  For dividing fractions, would it be okay to divide numerators and denominators straight across?  Explain why or why not. 
\begin{freeResponse}
\end{freeResponse}
\vfill
\end{problem}

%\vspace{2in}
%
%\begin{problem}
%Describe three different ways of thinking about division.
%\begin{freeResponse}
%\end{freeResponse}
%\end{problem}
%
\begin{problem}
What is $1 \frac{3}{4} \div \frac{1}{2}$?  Draw a picture that will help you solve the problem.
\begin{freeResponse}
\end{freeResponse}
\vfill
\end{problem}
%
%\newpage 
%\begin{problem}
%Which of the following are word problems for $\frac{3}{4} \div \frac{1}{2}$?  Draw pictures to illustrate the problems.  (Note: Pictures are not necessary in Ximera.)
%
%\begin{enumerate}
%\item Beth poured $\frac{3}{4}$ cup of cereal in a bowl.  The cereal box says that 1 serving is $\frac{1}{2}$ cup.  How many servings are in Beth's bowl? 
%
%\item Beth poured $\frac{3}{4}$ cup of cereal in a bowl.  Then Beth took $\frac{1}{2}$ of that cereal and put in into another bowl.  How many cups of cereal are in the second bowl? 
%
%\item A crew is building a road.  So far, the road is $\frac{3}{4}$ mile long.  This is $\frac{1}{2}$ the length that the road will be when it is finished.  How many miles long will the finished road be? 
%\item If $\frac{3}{4}$ cup of flour makes $\frac{1}{2}$ a batch of cookies, then how many cups of flour are required for a full batch of cookies?  
%
%\item If $\frac{1}{2}$ cup of flour makes 1 batch of cookies, then how many batches of cookies can you make with $\frac{3}{4}$ cup of flour?  
%
%\item If $\frac{3}{4}$ cup of flour makes 1 batch of cookies, then how much flour is in $\frac{1}{2}$ of a batch of cookies? 
%
%\end{enumerate}
%(Note:  These problems are borrowed from Sybilla Beckmann's book, \emph{Mathematics for Elementary Teachers}.)
%\begin{freeResponse}
%\end{freeResponse}
%
%\end{problem}



\end{document}
